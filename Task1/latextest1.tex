\documentclass[a4paper,10pt]{article}
\usepackage[utf8x]{inputenc}
\usepackage{graphicx}
\usepackage{fullpage}
\usepackage{listings}

\lstset{numbers=left, numberstyle=\tiny, stepnumber=1, numbersep=5pt, language=python,
showstringspaces=false, basicstyle=\small}

%opening
\title{Task1}
\author{Vogelman}

\begin{document}

\maketitle

\begin{abstract}
The results for task 5 which deals with Python components.
\end{abstract}

\section{Question 1}
Here should be your answer
etc. etc.
\begin{math}
B=\frac{1}{\pi\sqrt{a}}[0+x_{e}.\sqrt{a}\sqrt{\pi}\sqrt{\pi}-0]
\end{math}
%%$$
%% \frac{n!}{k!(n-k)!} = \binom{n}{k}
%%$$

% Let's start to include a figure (must be a pdf file)
\begin{figure}[h]
  \centering
  \includegraphics[width=6cm]{parabola.pdf}
  \caption{\it Example of a gaussian shaped source with fitted ellipse.
           The ellipse follows the contour for which the pixel values are half
           the maximum value. It has a major axis and a minor axis.
           The angle between the major axis and the x axis is the rotation
           angle $\theta$ of the ellipse. The position angle
           $\varphi$ is equal to $90^o+\theta$.}
  \label{fig:parabola}
\end{figure} 

\begin{figure}[h!]
  \centering
  \includegraphics[width=8cm]{ffts.png}
  \caption{\it Fast Fourier transforms}
  \label{fig:fft}
\end{figure} 



\begin{lstlisting}
#!/usr/bin/env python
import numpy
import pyplot

x = numpy.arange( 3,15, dtype='f')
y = 2.0 * x + 3.0

pyplot.plot(x,y)
pyplot.savefig( 'plot.v1.png' )
pyplot.show()
\end{lstlisting}


\end{document}
